\documentclass[a4paper, fontset=none, UTF8]{ctexbook}

\usepackage{fontspec}
  \setmainfont{Libertinus Serif}
  \setsansfont{Libertinus Sans}
  \setmonofont{Source Han Sans HW TC}
  % \setmonofont{Source Code Pro}
\usepackage{xeCJK}
  \setCJKmainfont{Source Han Serif TC}
  \setCJKsansfont{Source Han Sans TC}
  \setCJKmonofont{Source Han Sans HW TC}

\usepackage{xcolor}
\usepackage[hidelinks]{hyperref}

\title{製作音樂:電子音樂製作人的74個創意策略}
\author{Dennis DeSantis}

\begin{document}
\maketitle
\tableofcontents
\clearpage
\begin{center}
  獻給 Alison 和 Cecilia,\\
  你們激勵了我所做的一切。
\end{center}
\clearpage
\chapter{序言}
\textbf{這本書講了甚麼?}

對於很多藝術家而言,除了創作藝術以外,沒有甚麼更能引發對存在感的恐懼了。那種自己不夠好或者自己知識不足的恐懼導致了無數創作上的危機並使的許多潛在的傑作無法問世。

電子音樂人過去總是可以找技術的笨重難用和不成熟當作自己不作為的藉口,但是如今,音樂人處在工具與技術的黃金時代。一款僅需99美分的智慧型手機應用程式,就能提供給你價值百萬的錄音室的功能;一首新歌一創作完成就可立刻分享至全球;透過Google搜索,各種音效設計與音樂製作技巧的教學皆唾手可得。這些發展已然徹底改變了音樂人的競爭環境,並且使得臥室製作人\footnote{譯註:臥室製作人(bedroom producer)}也得以創作出過去只有大唱片公司的藝術家才能達到水準的音樂。

即便如此,製作音樂還是很艱難的事,為什麼?
\end{document}
