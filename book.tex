\documentclass[a4paper, fontset=none, UTF8]{ctexbook}

\usepackage{fontspec}
  \setmainfont{Libertinus Serif}
  \setsansfont{Libertinus Sans}
  \setmonofont{Source Han Sans HW TC}
  % \setmonofont{Source Code Pro}
\usepackage{xeCJK}
  \setCJKmainfont{Source Han Serif TC}
  \setCJKsansfont{Source Han Sans TC}
  \setCJKmonofont{Source Han Sans HW TC}
  \renewcommand{\familydefault}{\sfdefault}

\usepackage{tikz} % 圖案繪製
\usepackage{eso-pic} % 頁面背景圖案
\usepackage{xcolor} % 顏色定義
\usepackage{enumitem} % 列表樣式設定
  \setlist[itemize]{label={>} , leftmargin=2em, nosep, itemsep=0em}
\usepackage[hidelinks]{hyperref} % 超連結

% region 主內容背景樣式設定
\definecolor{themeOrange}{RGB}{248, 166, 145}
\definecolor{themeTextOnOrange}{RGB}{0, 0, 0}
\definecolor{themeWhite}{RGB}{255, 255, 255}
\definecolor{themeTextOnWhite}{RGB}{0, 0, 0}
\definecolor{themeBlack}{RGB}{0, 0, 0}
\definecolor{themeTextOnBlack}{RGB}{255, 255, 255}
\newif\ifInProblemMode
\InProblemModefalse
\newif\ifInSolutionMode
\InSolutionModefalse
% \newcommand{\activeThemeColor}{}
\NewDocumentCommand{\activeThemeColor}{}{}
\AddToShipoutPictureBG{
  \begin{tikzpicture}[remember picture, overlay]
    \ifInProblemMode
      \fill[\activeThemeColor](current page.south west) rectangle (current page.north east);
      \ifodd\value{page}
        \fill[white]([xshift=-8mm] current page.north east) rectangle (current page.south east);
      \else
        \fill[white](current page.north west) rectangle ([xshift=8mm] current page.south west);
      \fi
    \fi
    \ifInSolutionMode
      \fill[white](current page.south west) rectangle (current page.north east);
      \ifodd\value{page}
        \fill[\activeThemeColor]([xshift=-8mm] current page.north east) rectangle (current page.south east);
      \else
        \fill[\activeThemeColor](current page.north west) rectangle ([xshift=8mm] current page.south west);
      \fi
    \fi
  \end{tikzpicture}
}
% endregion

% \newcommand{\CJKdash}{{\addCJKfontfeatures{CharacterWidth=Full}⸺}}
\NewDocumentCommand{\CJKdash}{}{{\addCJKfontfeatures{CharacterWidth=Full}⸺}}
% \newenvironment{myproblem}[3][Black]{
\NewDocumentEnvironment{myproblem}{O{Black}mm}{
    \InProblemModetrue\InSolutionModefalse
    \renewcommand{\activeThemeColor}{theme#1}\color{themeTextOn#1}
    % \RenewDocumentCommand{\activeThemeColor}{}{theme#1}\color{themeTextOn#1}
    \chapter{#2}
    \noindent\textbf{\huge 問題:}

    {\Large #3}

  }{\clearpage\InProblemModefalse}
% \newenvironment{mysolution}{
\NewDocumentEnvironment{mysolution}{O{Black}}{
    \InSolutionModetrue\InProblemModefalse
    \renewcommand{\activeThemeColor}{theme#1}
    % \RenewDocumentCommand{\activeThemeColor}{}{theme#1}
    \noindent\null\textbf{解決方案:}

  }{\clearpage\InSolutionModefalse}

\title{創作音樂:電子音樂製作人的74個創意策略}
\author{Dennis DeSantis}

\begin{document}
\maketitle

\begin{center}
  獻給 Alison 和 Cecilia,\\
  你們激勵了我所做的一切。
\end{center}
\clearpage

\chapter{序言}

\textbf{這本書講了甚麼?}

對於很多藝術家而言,除了創作藝術以外,沒有甚麼更能引發對存在感的恐懼了。那種自己不夠好或者自己知識不足的恐懼導致了無數創作上的危機並使的許多潛在的傑作無法問世。

電子音樂人過去總是可以找技術的笨重難用和不成熟當作自己不作為的藉口,但是如今,音樂人處在工具與技術的黃金時代。一款僅需99美分的智慧型手機應用程式,就能提供給你價值百萬的錄音室的功能;一首新歌一創作完成就可立刻分享至全球;透過Google搜索,各種音效設計與音樂製作技巧的教學皆唾手可得。這些發展已然徹底改變了音樂人的競爭環境,並且使得臥室製作人\footnote{譯註:臥室製作人(bedroom producer)指的是在家中(通常在臥室)利用自己的電腦、軟體、樂器、其他設備等有限資源獨立完成音樂創作的人,此類人大多屬於業餘人員。}也得以創作出過去只有大唱片公司的藝術家才能達到水準的音樂。

即便如此,製作音樂還是很艱難的事,為什麼?

《創作音樂》正是為了解答這個問題,並同時提供一條簡化創作的途徑。本書系統性地闡述了一套具體的模式,可供音樂創作時運用,從而推進創作的進程。

每種模式皆以以下方式表述:

\begin{itemize}
  \item 提出問題。所謂的問題,就是阻礙你推進特定樂曲進度的障礙。本書中的問題皆源自真實情境\CJKdash{}你很可能會發現其中許多正是過去曾阻礙你的因素。問題可能出現在創作初期(例如不知如何開頭)、中期(例如創作大量素材卻不知如何組織)或接近尾聲時(例如不斷修改卻無法決定如何收尾)。
  \item 透過實例,有時輔以其他模式參考,來深入解析問題。
  \item 提供解決方案。方案是能化解特定困境的實用步驟或指令組合。和問題一樣,這些方案也都來自真實情境。只要實踐這些方案,問題將迎刃而解。不過請注意,這也同時要求你切實實踐這些方案,在絕大多數情況下,紙面上的閱讀是無法推動你向前的。
  \item 透過實例,有時輔以其他模式參考,來深入解析方案。
\end{itemize}

\textbf{此書面向誰?}

若你在使用電腦創作原創音樂,並且為如何完成音樂專案而感到苦惱,那《創作音樂》就是為你而寫的。雖然書中探討的許多模式可調整應用於其他音樂創作類型(例如為搖滾樂團或弦樂四重奏譜曲),但此書的目標是為了解決那些和機器設備打交道的人遇到的特定問題,而並非和樂器打交道的或者其他人士。

雖然在借鑑此書時並不需要任何的先備技能,不過我在撰寫時仍假設讀者對某種數位音訊工作站(DAW)或者類似的音樂製作環境具有一些基礎認識。本書並不要求特定的工具,書內所探討的問題和解決方案亦並不依賴任何特定技術的操作流程。如你對樂理有一些基礎的知識\CJKdash{}類如和絃、音階或者是節奏\CJKdash{}也將會是有助益的,不過這同樣也不是必需的。

儘管這這些模式並非總是那樣顯而易見,但它們都是有足過的普適性可使得任何電子音樂流派的音樂人加以運用,無論是商業舞曲還是前衛音樂都是如此。雖然部分解說會援引實際的流派甚至是具體的樂曲範例,但我鼓勵你嘗試跳出這些描述,以專注於模式的本質,如此以來便可以使其靈活應用至自身的創作之中。

\textbf{此書的作者是誰?}

我是一位有著多元背景的音樂人。我曾研習過古典作曲、樂理以及打擊樂,不過如今我的主要創作是以house和techno為方向的電子音樂。我成長於底特律的郊區,而這座城市所產出並傳遞的音樂深深影響了我早期的音樂發展。即便此書力求不局限於具體流派,但內容仍會有大量我個人的觀點視角,書中所探討的皆是我在自身音樂實踐中思考和運用的反映。

\textbf{如何運用本書?}

試試看把《創作音樂》看成是旅行手冊。書中的模式之間並沒有明確的序列關係,不過內容會大致上依照概念來鬆散分組。你可以按自己在創作途中遇到的具體問題來按需閱讀並實驗各種模式。有時模式之間會有明確的關聯關係,而且我也常常援引其他的模式來闡明當前的內容。因此,雖然並不會要求說需要從頭至尾通讀全書,不過完整全面閱讀會對發現各個模式之間的關聯有所助益,能更好的將它們看做是一個系統整體而不是一個各地孤立案例。

\textbf{此書是如何組織的?}

這些模式本身是根據特定音樂上的問題在創作過程當中可能出現在的階段來分組的。其包括了:
\begin{itemize}
  \item 起始時的問題。這些問題會完全阻礙你,導致你無法開始創作,這類問題包括有缺乏靈感、無法將腦中構想的聲音實現出來等。解決方案方面則包括培養更好主動聆聽能力的練習,以及探索各種玩轉聲音、和聲、旋律、節奏以及音樂形式的方法。
  \item 過程中的問題。這是創作中最常見到的障礙,一般發生在完成了某些部分但是離結束尚早的時侯,問題包括了疲勞、素材的開發與變化等。解決方案包括有對產生新素材的練習、歌曲結構的塑造、創作流動性的維持。
  \item 收尾時的問題。當你覺得一切近乎完成(但還是差一點點),但是又沒法滿意收官的時候便會遇到此方面的問題。解決方案包括有創造果敢編排和有力結尾的一些主意。
\end{itemize}

\textbf{為何需用此書?}

包括書籍、課程、影片教學、軟體文件和私人教學老師在內,學習音樂技術和音樂製作的途徑早已是數不勝數。我也同樣極力支持這些資源並且鼓勵對這本書感興趣的讀者善用這些途徑。不過所有的這些資源都幾乎全部聚焦在技術或者製作這方面,而不是在音樂本身這方面。《創作音樂》正是為那些已經在技術層面已經掌握了基本的音樂製作方法,但是仍對創作過程感到步履維艱的人(我猜這適用於你我在內的所有人!\null{})所寫的。

\textbf{最後……}

這本書不會教你如何使用壓縮器、可編碼合成器或者是製出完美的鼓聲,這類的音樂製作技巧已有大量的專門書籍詳細描述。本書將做的事情是引導你運用這些工具去創作,並且特別會著重於解決音樂難題和持續進步,以及完成你的創作(這也是最為關鍵的部分)。

我期待《創作音樂》可以為你帶來啟發,但我更希望能夠啟發你的是你運用這些模式所創作的出的音樂本身。《創作音樂》並非是泛泛其談的格言合輯,其真正意義在於將關於音樂創作哲學與心理學的激勵理念與各類音樂人的實用工具技巧相結合,以此助你真正完成創作。

\tableofcontents

\part{起始時的問題}

\begin{myproblem}[Orange]{三種起手方式}{You’re staring into the void of a new empty project in your DAW, and you have absolutely no idea how to begin.}
好!
\end{myproblem}
\begin{mysolution}[Orange]
壞。
\end{mysolution}

\chapter{其他}


\end{document}
