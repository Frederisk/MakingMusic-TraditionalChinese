\documentclass[a4paper,twoside,openany,fontset=none,UTF8]{ctexbook}

\usepackage{fontspec}
  \setmainfont{Libertinus Serif}
  \setsansfont{Libertinus Sans}
  \setmonofont{Source Han Sans HW TC}
  % \setmonofont{Source Code Pro}
\usepackage{xeCJK}
  \xeCJKsetup{PunctStyle=quanjiao}
  \setCJKmainfont{Source Han Serif TC}[Language=Chinese Traditional]
  \setCJKsansfont{Source Han Sans TC}[Language=Chinese Traditional]
  \setCJKmonofont{Source Han Sans HW TC}[Language=Chinese Traditional]
  \renewcommand{\familydefault}{\sfdefault}
  \xeCJKEditPunctStyle{quanjiao}{optimize-kerning=true}
  \ctexset{
    part/pagestyle=empty
  }

\usepackage[headheight=28pt]{geometry}
% \usepackage{multicol}
\usepackage{tikz} % 圖案繪製
\usepackage{musicography}
\usepackage{eso-pic} % 頁面背景圖案
\usepackage{xcolor} % 顏色定義
\usepackage{enumitem} % 列表樣式設定
  \setlist[itemize]{leftmargin=2em,nosep,itemsep=0em,label={>}}
  \setlist[enumerate]{leftmargin=2em,nosep,itemsep=0em,label={\arabic*.}}
\usepackage[hidelinks,pdfusetitle]{hyperref} % 超連結
\usepackage{fancyhdr,extramarks}
  \fancyhf{}
  \renewcommand{\partmark}[1]{\markboth{#1}{}}
  \renewcommand{\chaptermark}[1]{\markright{#1}}
  \renewcommand{\headrulewidth}{0pt}
  \fancyhead[LE,RO]{\small{\color{\lastleftxmark}\leftmark}\\\rightmark}
  \extramarks{black}{ignored}
  \fancyfoot[LE,RO]{\thepage}
  \fancypagestyle{plain}{
    \fancyhf{}
    \fancyfoot[LE,RO]{\thepage}
  }

% region 主內容背景樣式設定
\definecolor{themeOrange}{RGB}{248,166,145}
\definecolor{themeTextOnOrange}{RGB}{0,0,0}
\definecolor{themeWhite}{RGB}{255,255,255}
\definecolor{themeTextOnWhite}{RGB}{0,0,0}
\definecolor{themeBlack}{RGB}{0,0,0}
\definecolor{themeTextOnBlack}{RGB}{255,255,255}
\newif\ifInProblemMode
\InProblemModefalse
\newif\ifInSolutionMode
\InSolutionModefalse
% \newcommand{\activeThemeColor}{}
\NewDocumentCommand{\activeThemeColor}{}{}
\AddToShipoutPictureBG{
  \begin{tikzpicture}[remember picture,overlay]
    \ifInProblemMode
      \fill[\activeThemeColor](current page.south west) rectangle (current page.north east);
      \ifodd\value{page}
        \fill[white]([xshift=-8mm] current page.north east) rectangle (current page.south east);
      \else
        \fill[white](current page.north west) rectangle ([xshift=8mm] current page.south west);
      \fi
    \fi
    \ifInSolutionMode
      \fill[white](current page.south west) rectangle (current page.north east);
      \ifodd\value{page}
        \fill[\activeThemeColor]([xshift=-8mm] current page.north east) rectangle (current page.south east);
      \else
        \fill[\activeThemeColor](current page.north west) rectangle ([xshift=8mm] current page.south west);
      \fi
    \fi
  \end{tikzpicture}
}
% endregion

% \newcommand{\CJKdash}{{\addCJKfontfeatures{CharacterWidth=Full}⸺}}
\NewDocumentCommand{\CJKdash}{}{{\addCJKfontfeatures{CharacterWidth=Full}⸺}}
% \newenvironment{myproblem}[3][Black]{
\NewDocumentEnvironment{myproblem}{O{Black}mm}{
    \InProblemModetrue\InSolutionModefalse
    \renewcommand{\activeThemeColor}{theme#1}\extramarks{white}{ignored}\color{themeTextOn#1}
    % \RenewDocumentCommand{\activeThemeColor}{}{theme#1}\color{themeTextOn#1}
    \chapter{#2}
    \noindent\textbf{\huge 問題:}

    {\Large #3}

  }{\clearpage\InProblemModefalse\extramarks{black}{ignored}}
% \newenvironment{mysolution}{
\NewDocumentEnvironment{mysolution}{O{Black}}{
    \InSolutionModetrue\InProblemModefalse
    \renewcommand{\activeThemeColor}{theme#1}\extramarks{\activeThemeColor}{ignored}
    % \RenewDocumentCommand{\activeThemeColor}{}{theme#1}
    \noindent\null\textbf{解決方案:}

  }{\clearpage\InSolutionModefalse\extramarks{black}{ignored}}

\title{創作音樂:電子音樂製作人的74個創意策略}
\author{Dennis DeSantis}

\begin{document}
\pagestyle{empty}
\maketitle

\clearpage
\null\vfill
{\setlength{\parindent}{0pt}\setlength{\parskip}{6pt}
\hyperlink{https://www.ableton.com}{www.ableton.com}\\
保留所有權利。

Ableton AG 出版\\
Schönhauser Allee 6-7\\
10119 Berlin, Germany

原版書籍ISBN:978-3-9817165-0-4
}

\clearpage

\begin{center}
  獻給 Alison 和 Cecilia,\\
  你們激勵了我所做的一切。
\end{center}
\clearpage

\chapter*{序言}\thispagestyle{empty}\addcontentsline{toc}{chapter}{序言}
% \frontmatter\chapter{序言}
\clearpage

% \begin{multicols}{2}
% \raggedcolumns\null
%\null\vfill\columnbreak
\textbf{這本書講了甚麼?}

對於很多藝術家而言,除了創作藝術以外,沒有甚麼更能引發對存在感的恐懼了。那種自己不夠好或者自己知識不足的恐懼導致了無數創作上的危機並使的許多潛在的傑作無法問世。

電子音樂人過去總是可以找技術的笨重難用和不成熟當作自己不作為的藉口,但是如今,音樂人處在工具與技術的黃金時代。一款僅需99美分的智慧型手機應用程式,就能提供給你價值百萬的錄音室的功能;一首新歌一創作完成就可立刻分享至全球;透過Google搜索,各種音效設計與音樂製作技巧的教學皆唾手可得。這些發展已然徹底改變了音樂人的競爭環境,並且使得臥室製作人\footnote{譯註:臥室製作人(bedroom producer)指的是在家中(通常在臥室)利用自己的電腦、軟體、樂器、其他設備等有限資源獨立完成音樂創作的人,此類人大多屬於業餘人員。}也得以創作出過去只有大唱片公司的藝術家才能達到水準的音樂。

即便如此,製作音樂還是很艱難的事,為什麼?

《創作音樂》正是為了解答這個問題,並同時提供一條簡化創作的途徑。本書系統性地闡述了一套具體的模式,可供音樂創作時運用,從而推進創作的進程。

每種模式皆以以下方式表述:

\begin{itemize}
  \item 提出問題。所謂的問題,就是阻礙你推進特定樂曲進度的障礙。本書中的問題皆源自真實情境\CJKdash{}你很可能會發現其中許多正是過去曾阻礙你的因素。問題可能出現在創作初期(例如不知如何開頭)、中期(例如創作大量素材卻不知如何組織)或接近尾聲時(例如不斷修改卻無法決定如何收尾)。
  \item 透過實例,有時輔以其他模式參考,來深入解析問題。
  \item 提供解決方案。方案是能化解特定困境的實用步驟或指令組合。和問題一樣,這些方案也都來自真實情境。只要實踐這些方案,問題將迎刃而解。不過請注意,這也同時要求你切實實踐這些方案,在絕大多數情況下,紙面上的閱讀是無法推動你向前的。
  \item 透過實例,有時輔以其他模式參考,來深入解析方案。
\end{itemize}

%\vfill\null\columnbreak
\textbf{此書面向誰?}

若你在使用電腦創作原創音樂,並且為如何完成音樂專案而感到苦惱,那《創作音樂》就是為你而寫的。雖然書中探討的許多模式可調整應用於其他音樂創作類型(例如為搖滾樂團或弦樂四重奏譜曲),但此書的目標是為了解決那些和機器設備打交道的人遇到的特定問題,而並非和樂器打交道的或者其他人士。

雖然在借鑑此書時並不需要任何的先備技能,不過我在撰寫時仍假設讀者對某種數位音訊工作站(DAW)或者類似的音樂製作環境具有一些基礎認識。本書並不要求特定的工具,書內所探討的問題和解決方案亦並不依賴任何特定技術的操作流程。如你對樂理有一些基礎的知識\CJKdash{}類如和絃、音階或者是節奏\CJKdash{}也將會是有助益的,不過這同樣也不是必需的。

儘管這這些模式並非總是那樣顯而易見,但它們都是有足過的普適性可使得任何電子音樂流派的音樂人加以運用,無論是商業舞曲還是前衛音樂都是如此。雖然部分解說會援引實際的流派甚至是具體的樂曲範例,但我鼓勵你嘗試跳出這些描述,以專注於模式的本質,如此以來便可以使其靈活應用至自身的創作之中。

\textbf{此書的作者是誰?}

我是一位有著多元背景的音樂人。我曾研習過古典作曲、樂理以及打擊樂,不過如今我的主要創作是以house和techno為方向的電子音樂。我成長於底特律的郊區,而這座城市所產出並傳遞的音樂深深影響了我早期的音樂發展。即便此書力求不局限於具體流派,但內容仍會有大量我個人的觀點視角,書中所探討的皆是我在自身音樂實踐中思考和運用的反映。

\textbf{如何運用本書?}

試試看把《創作音樂》看成是旅行手冊。書中的模式之間並沒有明確的序列關係,不過內容會大致上依照概念來鬆散分組。你可以按自己在創作途中遇到的具體問題來按需閱讀並實驗各種模式。有時模式之間會有明確的關聯關係,而且我也常常援引其他的模式來闡明當前的內容。因此,雖然並不會要求說需要從頭至尾通讀全書,不過完整全面閱讀會對發現各個模式之間的關聯有所助益,能更好的將它們看做是一個系統整體而不是一個各地孤立案例。

\textbf{此書是如何組織的?}

這些模式本身是根據特定音樂上的問題在創作過程當中可能出現在的階段來分組的。其包括了:
\begin{itemize}
  \item 起始時的問題。這些問題會完全阻礙你,導致你無法開始創作,這類問題包括有缺乏靈感、無法將腦中構想的聲音實現出來等。解決方案方面則包括培養更好主動聆聽能力的練習,以及探索各種玩轉聲音、和聲、旋律、節奏以及音樂形式的方法。
  \item 過程中的問題。這是創作中最常見到的障礙,一般發生在完成了某些部分但是離結束尚早的時侯,問題包括了疲勞、素材的開發與變化等。解決方案包括有對產生新素材的練習、歌曲結構的塑造、創作流動性的維持。
  \item 收尾時的問題。當你覺得一切近乎完成(但還是差一點點),但是又沒法滿意收官的時候便會遇到此方面的問題。解決方案包括有創造果敢編排和有力結尾的一些主意。
\end{itemize}

\textbf{為何需用此書?}

包括書籍、課程、影片教學、軟體文件和私人教學老師在內,學習音樂技術和音樂製作的途徑早已是數不勝數。我也同樣極力支持這些資源並且鼓勵對這本書感興趣的讀者善用這些途徑。不過所有的這些資源都幾乎全部聚焦在技術或者製作這方面,而不是在音樂本身這方面。《創作音樂》正是為那些已經在技術層面已經掌握了基本的音樂製作方法,但是仍對創作過程感到步履維艱的人(我猜這適用於你我在內的所有人!)所寫的。

\textbf{最後……}

這本書不會教你如何使用壓縮器、可編碼合成器或者是製出完美的鼓聲,這類的音樂製作技巧已有大量的專門書籍詳細描述。本書將做的事情是引導你運用這些工具去創作,並且特別會著重於解決音樂難題和持續進步,以及完成你的創作(這也是最為關鍵的部分)。

我期待《創作音樂》可以為你帶來啟發,但我更希望能夠啟發你的是你運用這些模式所創作的出的音樂本身。《創作音樂》並非是泛泛其談的格言合輯,其真正意義在於將關於音樂創作哲學與心理學的激勵理念與各類音樂人的實用工具技巧相結合,以此助你真正完成創作。
% \end{multicols}

\tableofcontents\thispagestyle{empty}
\pagestyle{fancy}
{\pagecolor{themeOrange}\part{起始時的問題}}

\begin{myproblem}[Orange]{三種起手方式}{你正凝視著DAW中那個嶄新而又空蕩蕩的專案,但完全不知道如何開始著手。}
在創作環境中,空白一片的狀態或許是創作環境中最令人生畏的時候。在我們進入創作的狀態之中時,靈感往往接連湧出,但在毫無頭緒之時,所有的選擇都是一種可能,這反而導致你我難以抉擇。「一步步來」這一章中(第X頁)提到的無疑也是一種推進的方式,但推進應該從哪一步開始呢?當你一無所有的時候,即便是單單這一步也會變得像是無法逾越的障礙。

一個看似簡單的答案是:「怎樣開始無關緊要,關鍵在於行動!」這句話作為體育運動的口號而言或許十分適合,但對於創作性質的工作而言過於輕率敷衍,從而缺乏實際的指導意義\CJKdash{}畢竟在這個領域而言,可以直接行動去做的事物可以說是無窮無盡的,而且前進的路徑也不是那樣顯而易見。所以在這兒,我提供有三點可以使你從零開始的實用建議:
\end{myproblem}
\begin{mysolution}[Orange]
\begin{enumerate}
  \item \textbf{從基礎開始。}在多數音樂類型中,我們可以將類如貝斯、鼓組的低音或是純節奏樂器視作音樂的「基底」層,然後再在此基礎上向上逐步疊加音高更高的樂器而作成的。從底層開始構築可以為後續所有元素奠定概念和音樂基礎。鼓組通常承擔核心節拍功能,而貝斯則透過音符定義和錨定和弦進行。即便是著手於那些捨棄了傳統樂器的實驗性曲風,這種可視作基礎的元素仍會存在。比方說持續的低鳴聲層,或者是類如重複節奏的聲響。
  \item \textbf{從你聽到的開始。}許多音樂人沒有或者說幾乎沒有過即興的音樂靈感\CJKdash{}他們的所有音樂作品皆來自主動創作。若你幸運地在自己腦海中聽見了自己的原創音樂構想,務必將其作為自身創作的根基。這種構想舉例比如有自己反覆哼唱旋律的想法,或是在桌上敲出的鼓點節奏。這些靈感雖然並非是在創作途中產生,但完全不應該就因此而棄置。相反,這類偶然的靈光一現往往會是最有趣的創意想法。
  \item \textbf{從你所熟悉的開始。}如果你能演奏「真正的」實體樂器,那不妨用它來激發靈感。即便你所創作的純電子音樂,而且並不打算在作品中使用樂器的錄音,但你與樂器之間的自然的實體連結仍可能會幫助你構想出比單純用滑鼠和MIDI控制器更有趣味更鮮活的音樂構想。比方說,吉他手與鍵盤手在和聲以及和弦的處理方式上往往截然不同。不過由於鋼琴鍵盤已經算是DAW的事實性標準控制介面,這導致很多吉他手可能甚至從未想過在電子音樂創作中運用吉他。鼓手也很少會去實際演奏那些電子音樂中常見的節奏類型,但坐在真實鼓組後方實際體驗中所激發的創意靈感,往往是電子鼓墊控制器上所感受不到的。此方法的關鍵在於如何精準將聲學上的想法精準轉換至電子媒介上,這無疑是種幸福的煩惱\CJKdash{}相較於毫無靈感而言。
\end{enumerate}
\end{mysolution}
\begin{myproblem}[Orange]{特質編目}{富有創造了的音樂人會從其他音樂中汲取靈感。當我們試圖創作出獨具風格的音樂時,我們所聽到的每段音樂都會被自動處理,然後成為我們音樂詞彙潛意識中的一部分。從中挪取過多是偷竊,使用太少又會辜負影響來源。} % TODO: 亮點清單?
對於致力發掘並發展獨特聲音的音樂人而言,每當聆聽啟發性作品時,總會有個需化解的內在矛盾:要真正做到原創,是否必須摒棄所有外部影響?從他人的作品的借鑑到何種程度會讓作品失去自我創作的感覺?致敬/啟發與抄襲/剽竊之間的界線在哪裡? % TODO: 糟糕的段落。

不過遺憾的是,這些問題並沒有所謂的「正確」答案。先放過客觀上的法律問題不談,每位藝術家都需自行釐清在借鑒他人作品時能接受的適當度。不過確實也存在一些策略來既能讓你的創作融入靈感的「精隨」,又能強制你創作出的是新的作品。其中的一種方式就是寫一份特質編目。
\end{myproblem}
\begin{mysolution}[Orange]
仔細反覆地聆聽那首激發了你靈感的樂曲(也就是「源」),逐層逐個剖析直至你可以完整寫出一份它的特質編目。當清單趨向完備後,將原曲先放在一旁,並僅以此清單為模板來創作你自己的新作品(也即「目標」)。

考量一下聲音、和聲、旋律、節奏與曲式等的種種特質,針對每項特質具體描述出你所聽到的內容。如果你熟悉用一些樂譜記號的話也可在這個清單中適度運用,但要適度,因為你的目標是為了捕捉原始素材的脈絡和框架,包括那些令人受啟發的方面,但絕不是簡單去拷貝。你最終的得到的產出應當是描述性的文字,而不是單純把樂譜轉錄一遍。

這份清單的詳細程度取決於很多方面:將聽聞內容轉化為文字的能力、原始內容的深度與複雜性、為此投入的時間長短等等。這個清單的關鍵點不在於它具體程度到底有多少,而是它是否能夠提供足夠的內容供你自己運用,也就是在創作過程中不再需要再次參考原始的材料。

一份簡單的特質編目大致上會長這樣子:

\begin{itemize}
  \item 122 bpm
  \item 聲音元素:鼓組(808、四四拍、運用了大量濾波器的閉鈸聲效)、低音聲部(FM風格?)、電鋼琴(失真但乾爽)、女聲(有喘息聲的主歌、全聲量的副歌)、合成器主音(巨大的超鋸齒波\footnote{譯註:big supersaw}、但僅在第二段副歌后有出現)。
  \item 和聲:大量的D小調與A大調相交替一直到間奏。間奏部分使用了D大調(大體上是這樣?)。間奏之後曲子的其餘部分又轉用E小調與B大調相交替。
  \item 旋律:不是很豐富,大量的D音,偶爾會跳升至A或者降到B\musFlat{}。
  \item 節奏:四四拍鼓組(基本的house節拍)、低音聲部多為弱拍八分音符(trance類型音樂的影響?)。在每四小節的第二拍的「and」位置上加入了酷爽的金屬敲擊聲。% TODO:「and」位置?
  \item 曲式:層層堆疊;以鼓聲開場、然後每種元素依次入場。在過門中僅剩鈸聲和低音聲部,然後再層層堆疊重建。全曲以16、32小節組成的段落構成(主歌16小節、主歌16小節、副歌16小節、過門32小節、副歌16小節、副歌16小節)。
\end{itemize}

這份特質編目足以描述無數嶄新的作品,實際上你腦海中或許早已浮現符合這些特徵的樂曲。此清單作為創作範本而言已足夠完整,但又不會過於具體到能完全復現出任何現存樂曲的地步;如若兩位不同的音樂家閱讀此清單,他們也幾乎不可能依此寫出相同的作品。現在試試將自己置身於某一位音樂家的思緒之中;僅以這份清單作為食譜,創造出嶄新的作品。
\end{mysolution}

\begin{myproblem}[Orange]{迴避清單}{當我們回放剛完成的曲目,有時會發現它與過往的創作有些過於相像了,雖說我們已經發展出了個人風格,但卻又深受其束縛,以致難以創作出真正嶄新的東西。}
我們常在完成曲目數日後再次重聽時才會發現這點。在創作過程中(或者剛完成不久時)有時會察覺不到有相似重複的地方,但當距離有點時間之後再聽,這種與自己過往作品的相似之處就會突然痛楚顯露出來。相似通常會在聲音上顯露出來\CJKdash{}同類的取樣或合成器聲音不斷出現。有時則會是曲式上的\CJKdash{}我們在不同的曲目之間依賴於使用同樣的轉場方式或結構框架。這讓人感覺自己就像是在不斷創作相同的樂曲。

要注意,根據你所從事的音樂類型,這種高度的一致性也可以是一種預期目標。比方說對EDM等商業風格而言,結構和整體聲音的可預測性正是此風格的基本特徵。激進的聲音設計與非常規的曲式在此類音樂中並無意義,與此相反,其吸引點在於藝術家如何在定義該流派的嚴格框架內展現創作功力。

但若你不喜歡陷入無限循環的這種感受,這裡有幾個或許可行的解決方案。
\end{myproblem}
\begin{mysolution}[Orange]
一旦
\end{mysolution}

\end{document}
